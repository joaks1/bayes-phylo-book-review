%&<latex>
\documentclass[letterpaper,12pt]{article}

\pdfpagewidth = 8.5in
\pdfpageheight = 11.0in
\usepackage[left=1in,right=1in,top=1in,bottom=1in]{geometry}
\usepackage{setspace}
\usepackage{xspace}
\usepackage{authblk}

\pagestyle{plain}
\pagenumbering{arabic}

\usepackage[round]{natbib}
\usepackage{hyperref}
% \hypersetup{pdfborder={0 0 0}, colorlinks=true, urlcolor=black, linkcolor=black, citecolor=black}
\hypersetup{pdfborder={0 0 0}, colorlinks=true, urlcolor=blue, linkcolor=blue, citecolor=blue}

\newcommand{\highLight}[1]{\textcolor{magenta}{\MakeUppercase{#1}}}
\newcommand{\ignore}[1]{}
\newcommand{\super}[1]{\ensuremath{^{\textrm{#1}}}}
\newcommand{\sub}[1]{\ensuremath{_{\textrm{#1}}}}
\newcommand{\dC}{\ensuremath{^\circ{\textrm{C}}}}
\newcommand{\booktitle}{\textit{Bayesian Phylogenetics}\xspace}
\newcommand{\booksubtitle}{\textit{Methods, Algorithms, and Applications}\xspace}
\newcommand{\bookfulltitle}{\textit{\booktitle: \booksubtitle}\xspace}
\newcommand{\editors}{Ming-Hui Chen, Lynn Kuo, and Paul O.\ Lewis\xspace}

\title{Book review: \bookfulltitle, edited by \editors}

\author[1,2]{Jamie R.\ Oaks\thanks{Corresponding author: \href{mailto:joaks@auburn.edu}{\tt joaks@auburn.edu}}}

\affil[1]{Department of Biology, University of Washington, Seattle, Washington 98195}
\affil[2]{Department of Biological Sciences, Auburn University, Auburn, Alabama 36849}

\date{\today}

%%%%%%%%%%%%%%%%%%%%%%%%%%%%%%%%%%%%%%%%%%%%%%%%%%%%%%%%%%%%
%%%%%%%%%%%%%%%%%%%%%%%%%%%%%%%%%%%%%%%%%%%%%%%%%%%%%%%%%%%%

\begin{document}

\maketitle

\newpage
\doublespacing

After the publication of Felsenstein's seminal paper \citep{Felsenstein1981},
phylogenetics matured into a likelihood-based statistical endeavor.
Thanks to this shift in perspective, phylogenetics has grown from a highly
specialized field into the statistical foundation for comparative biology.
However, this growth has not been easy.
Given the complex, high-dimensional nature of phylogenetic models
\citep{Kim2000}, phylogenetic inference is a particularly challenging
statistical problem, both theoretically and computationally. 
As models of diversification and character change have become richer to better
capture processes of biological evolution, the appeal of Bayesian approaches to
tackle the challenges of phylogenetic inference has greatly increased.
Bayesian approaches can leverage numerical tricks (e.g., data augmentation) and
prior information to make inference under richer evolutionary models practical.

Since the first Bayesian approaches to phylogenetic inference were introduced
\citep{Rannala1996,Mau1997} almost 20 years ago, Bayesian phylogenetics has
rapidly increased in sophistication and popularity.
In fact, the paper describing the most popular Bayesian phylogenetic software
\citep{Ronquist2003} is listed as the 100\super{th} most cited scientific paper
of all time \citep{VanNoorden2014}.
Despite this success, the literature has lacked a book devoted to the field of
Bayesian phylogenetics.
\bookfulltitle, edited by \editors \citep{Chen2014}, fills that void.

\booktitle is a collection of 12 chapters contributed by leaders in the field.
This book is not meant to be a comprehensive or gentle introduction to Bayesian
phylogenetics.
Rather, in the editors' own words, ``this book provides a snapshot of current
research in Bayesian phylogenetics, and was envisioned as a way of bringing
state-of-the-art phylogenetics to the attention of the Bayesian statistical
community, and state-of-the-art Bayesian statistics to the attention of the of
the phylogenetics community, with the ultimate goal of encouraging further
interdisciplinary research.''
As a snapshot of the current trends in Bayesian phylogenetics, this book is
missing many areas of active research within the field.
However, for the topics that are covered, including model selection, priors,
more efficient numerical algorithms, more realistic evolutionary models,
phylodynamics, recombination, and divergence-time estimation, 
the book successfully represents the cutting-edge of current research.
Overall, for a reader with the necessary background in statistics, phylogenies,
and evolutionary models, \booktitle is a great resource to get up to speed with
the current state and future directions of many areas of research in Bayesian
phylogenetics.

A strong emphasis of the book is Bayesian model choice, with five chapters
(Chapters 3--6, 9) making it their primary focus.
Given this emphasis, a meaningful review of the book is not possible without a
light (and very informal) introduction to marginal likelihoods and the
challenges of computing them.
The marginal likelihood of a model is the probability of the data under the
model, averaged, with respect to the prior, over the whole parameter space.
% It is also the normalizing constant of the posterior distribution.
Given the large number of parameters in phylogenetic models that need
to be summed or integrated, this marginalized measure of model fit cannot
be computed analytically.
In fact, the success of Bayesian phylogenetics up to this point is largely due
to numerical approaches (e.g., MCMC) that avoid ever having to calculate the
marginal likelihood.
However, as the diversity and richness of phylogenetic models continues to
rapidly increase, so does the need to compare the fit of models;
we can no longer avoid computing the marginal likelihood.

The simplest numerical approximation of the marginal likelihood is to take
samples of the model's parameters from the prior and turn the intractable
integral into a sum of the samples' likelihoods, remembering to correct for the
weight of the samples to keep everything proper. Because the weight of each
sample is one in this case, this simplifies to the average likelihood of the
prior samples.
Almost as simple, if we have a sample of the parameters from the posterior
distribution, we can again sum the likelihood of each sample.
The weight of each sample is not one in this case, but rather the ratio of the
prior density to the unnormalized posterior density.
Thus, the sum simplifies to the harmonic mean (HM) of the likelihoods of the
posterior sample \citep{Newton1994}.
Both of these importance-sampling approximations suffer severely from the fact
that the prior and posterior are often \emph{very} divergent from one another
(with the latter often \emph{much} more peaked than the former).
As a result, a finite sample from the prior is likely to miss the region of
parameter space with high likelihood, and thus often underestimate the marginal
likelihood.
Similarly, a finite sample from the posterior is likely to have very few
samples outside of the region of high posterior density, where the prior has a
strong downward ``pull'' on the average likelihood. This fails to penalize for
parameters and often results in overestimates of the marginal likelihood.

There are two ways to mitigate the problem caused by the sharp difference
between the prior and posterior.
One is to take many small steps, sampling from a series of power posterior
distributions, along the path connecting the unnormalized posterior to the
prior (i.e., turn a giant leap into many small steps);
the path-sampling (PS) \citep{Lartillot2006} and stepping-stone (SS)
\citep{Xie2011} methods use this strategy.
The second option is to not use the prior as a reference distribution, but
rather a different proper probability distribution that is as similar as
possible to the posterior (i.e., turn a giant leap into a smaller leap); 
generalized harmonic mean (GHM) estimators \citep{Gelfand1994} can use
this approach.
The former strategy requires a lot of extra computation that can be avoided by
the latter, but approaches that use the latter strategy tend to have greater
error.
The generalized stepping-stone (GSS) technique \citep{Fan2011} combines both
strategies, sampling from a series of power posteriors along the path
connecting the unnormalized posterior to a proper reference distribution that
is ideally as similar to the posterior as possible.
% To date, the GSS is the most efficient method for approximating marginal
% likelihoods of phylogenetic models.

The GSS was restricted to estimating marginal likelihoods for a fixed tree
topology, because a reference distribution for trees was lacking.
In Chapter 5, Holder et al.\ introduce a reference distribution on trees that
is parameterized by a sample of trees from the posterior.
A proper probability distribution of trees that is similar to the posterior
promises to be broadly applicable beyond the context of the GSS method.
For example, it can easily be used for generalized versions of HM (GHM) and PS
(GPS) techniques.
Arima and Tardella, allude to this by implementing fixed-tree GPS and GHM
methods in Chapter 3.


Arima and Tardella (Chapter 3) explore the performance of the inflated density
ratio (IDR) approach for estimating marginal likelihoods of phylogenetic models
\citep{Arima2012}.
The IDR is a variant of the GHM estimator where the use of the prior in the HM
estimator is replaced by a perturbation of the unnormalized posterior density.
They compared the IDR to the HM, GHM, SS, GSS, PS, and GPS techniques,
using simulated and biological data.
They find that the GSS and GPS are the best-performing estimators of the
marginal likelihood.
The GHM and IDR methods both perform much better than the HM estimator, thus
providing viable options when the computational burden of PS and SS approaches
is not practical.

Baele and Lemey (Chapter 4) assess the ability of various methods to accurately
choose the true relaxed-clock model from simulated sequence data.
They compare HM estimators, AICM (which estimates the Akaike information criterion
from the likelihoods of posterior samples), and PS and SS techniques.
Furthermore, they compare all of these marginal likelihood approximations to
the maximum \emph{a posteriori} model estimated by Bayesian model averaging via
Markov chain Monte Carlo (MCMC).
They find that PS, SS, and MAP perform similarly and better than AICM and HM
estimators.
% Baele and Lemey also use the SS approach to compare the fit of demographic
% models to sequences from HIV viruses.
% They find the marginal likelihood estimates from multiple demographic models
% are too similar to be confident in the best model, given the variance in SS
% estimates.

Prior to this book,
the IDR estimator was restricted to a fixed tree, because a perturbation of the
posterior distribution over trees was not available.
Wu et al. (Chapter 6) extend the IDR method to marginalize over trees.
They also prove the statistical consistency of the HM, GSS, and IDR estimators
when marginalizing over trees.
Using an empirical dataset, they also show that IDR performs much better than
the HM estimator, and provides a computationally efficient alternative to the
most accurate, yet computationally demanding, SS and PS methods.

Palczewski and Beerli (Chapter 9) derive an identity showing that the overall
marginal likelihood over an arbitrary number of independent subsets of 
data (e.g., loci in population genetic models) can be calculated from
the marginal likelihoods of each subset. 
This enables the marginal likelihood to be calculated from large-scale
datasets, by distributing the computation for each locus across many
processors.

Beyond the topic of estimating marginal likelihoods, \booktitle has several
other interesting chapters.
Wang and Yang (Chapter 2) provide a good introduction to priors in Bayesian
phylogenetics that is properly placed at the beginning of the book.
They give a succinct introduction to objective versus subjective perspectives
on Bayesian priors, but maintain the focus of the chapter on evaluating the
sensitivity of the posterior to prior assumptions.
They use a simple example of estimating the distance between between two DNA
sequences as a platform to introduce uniform, Jeffreys, and reference priors.
They also demonstrate how the use of independent priors for the large number of
branch-length parameters in phylogenetic models can cause problems.
Wang and Yang use an empirical example to show that even when these independent
branch-length priors are seemingly diffuse, they create a very strong prior on
the sum of these parameters (i.e., the length of the tree), which can lead to a
posterior that is biologically unreasonable.
Furthermore, they show that using a multivariate prior distribution on branch
lengths can avoid this problem.


Cheon and Liang (Chapter 7) and Bouchard-C\^{o}t\'{e} (Chapter 8) highlight
numerical integration algorithms that promise to complement the standard
workhorse, MCMC, for approximating posteriors of phylogenetic models.
After a very thorough review of numerical algorithms used in Bayesian
phylogenetics, Cheon and Liang introduce a sequential stochastic approximation
Monte Carlo (SSAMC) \citep{Liang2007,Cheon2008} algorithm.
Using simulated data, they show that SSAMC can avoid getting trapped in local
regions of high posterior density, while exhibiting smaller per-iteration CPU
times compared to popular, albeit outdated, MCMC implementations.
Bouchard-C\^{o}t\'{e} (Chapter 8) provides a very intuitive treatment of
sequential Monte Carlo (SMC) algorithms for approximating posteriors of trees.
Highlighting the "embarrassingly parallel" nature of SMC, he discusses
the computational gains SMC can offer via parallelization.
Bouchard-C\^{o}t\'{e} also discusses how the sequential nature of the
algorithm may allow it to accommodate richer models of sequence evolution
that can cripple MCMC.

Kuhner (Chapter 10) gives an intuitive overview of the extreme challenges
posed by sampling the space of ancestral recombination graphs (ARG).
While the difficult nature of the problem has made progress on estimating
ARGs slow, she highlights opportunities for future progress, which is ever more
important as biologists realize that reticulate processes of evolution
often should not be ignored.

Palacios et al.\ (Chapter 11) reviews nonparametric Bayesian methods of
inferring the effective size of populations through time. By framing
the problem as estimating the intensity function of a point process,
they discuss how the rich literature of computational techniques
for point processes provides avenues for continuing to move
this field forward.
Using simulated data, they also show how increasing the number of loci can lead
to more precise estimates of the effective population size through time.

Currently, phylogenetics is largely restricted to models of sequence evolution
with likelihood functions that can be feasibly calculated via Felsenstein's
pruning algorithm \citep{Felsenstein1981}.
To potentially circumvent this limitation, Hobolth and Thorne (Chapter 12)
propose Bayesian algorithms that can exploit the simpler likelihood form
of a fully observed Markov process.
Not only are such methods exciting due to their potential to make biologically
richer models of evolution computationally feasible, 
but also because they permit inferences of the detailed evolutionary history
of sequences along the tree.

In the last chapter of the book (Chapter 13),
Heath and Moore provide a very thorough and intuitive review of Bayesian
methods for inferring time-relative or time-absolute ultrametric phylogenies,
without assuming a constant rate of character change across the tree.
They present a useful categorization of Bayesian ``relaxed-clock'' models,
which aids in highlighting similarities and differences among them.

All of the chapters are important contributions to phylogenetics, but
\booktitle lacks cohesiveness as a book overall.
For example, the book lacks an introduction to phylogenetics that would be
helpful to orient readers without a strong background in the field.
The thorough review of Bayesian approaches to phylogenetics in Chapter 7 would
have been better placed as a separate chapter at the beginning of the book.
Also, given the amount of overlap among many of the chapters, an effort to
better maintain consistency of notation would have made it easier to flip back
and forth between chapters.
Furthermore, consistent divulging of the location of data and code would have
made the important contributions in the book more accessible to the broader
community of researchers in the field.
Perhaps, an auxiliary website with such information could become a useful,
albeit post hoc, complement to the book.

Despite the lack of cohesiveness, the medium of an edited volume allowed the
authors to write less formally and provide more background than would have been
possible in peer-reviewed papers.
Many of the authors took advantage of this, making difficult material much more
accessible and intuitive to the reader.

I expect \booktitle will be an important resource as Bayesian approaches to
phylogenetics continue to advance and diversify.
It introduces many novel methods and proofs that will be of broad interest to
statisticians, and includes thorough reviews of several exciting areas of
statistical phylogenetics.
As a result, many of the book's chapters will be important (and likely highly
cited) references.

\bibliographystyle{sysbio}
\bibliography{references}

\end{document}

